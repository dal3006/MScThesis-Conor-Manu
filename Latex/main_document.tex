
%% begin preamble%%

% set the document class
\documentclass[12pt]{report}

\usepackage[utf8]{inputenc}
\usepackage[T1]{fontenc}

%package to create a list of acronyms which appears in the table of contents
\usepackage[acronym, toc, nopostdot, nogroupskip]{glossaries}

% acronym list entries:
\newacronym{mbb}{MBB}{moving block bootstrap}
\newacronym{rss}{RSS}{residual sum of squares}
\newacronym{mae}{MAE}{mean absolute error}
\newacronym{rmse}{RMSE}{root mean squared error}
\newacronym{ann}{ANN}{Artificial Neural Network}
\newacronym{dl}{DL}{Deep Learning}
\newacronym{mse}{MSE}{mean squared error}
\newacronym{lstmnn}{LSTM NN}{Long Short Term Memory neural network}

% end acronym list entries

\makenoidxglossaries 

% aligns the description entries in the list of acronyms
\glsfindwidesttoplevelname
\setglossarystyle{alttree}

% setting the space between acronym and description
\renewcommand\glstreepredesc{\qquad} 

 
% prepare the document for images
\usepackage{graphicx}
% sets the directory where images are stored
\graphicspath{ {images/} }

% set spacing package to change the spacing for the table of contents
\usepackage{setspace}

% package to create captions and subcaptions for figures and tables
\usepackage[labelfont=bf]{caption}
\usepackage{subcaption}

% package to include notes underneath figures/tables
\usepackage{floatrow}

% package to type piecewise functions
\usepackage{amsmath}

% color package for formatting URLs 
\usepackage[dvipsnames]{xcolor}

% packages for cross references within the Latex document
\usepackage{varioref}

% package to create hyperlinks within the Latex document (including the bibliography if url is specified)

\usepackage[backref=true]{hyperref}

% the cleverref package   enhances   LATEX’s   cross-referencing   features,allowing the format of cross-references to be determined automatically ac-cording  to  the  “type”  of  cross-reference  (equation,  section,  etc.)   and  thecontext in which the cross-reference is used
\usepackage[capitalise, noabbrev]{cleveref}

% formatting the URLs
\hypersetup{colorlinks=true, linkcolor=BrickRed, urlcolor=RoyalBlue, citecolor=BlueViolet}
\urlstyle{same}


% using the geometry package to set margins of the document

\usepackage[a4paper,width=150mm,top=25mm,bottom=25mm]{geometry}

% creating the option to display bookmarks in the compiled pdf document
\usepackage{bookmark}

% using packages to add in customized headers and footers

\usepackage{fancyhdr}
\pagestyle{fancy}

% separation lines for header and footer


% specifying inputs for headers and footers (commented out for now)


\fancyhead[RO,LE]{Chapter \thechapter}
\fancyfoot[C]{\thepage}


% changing the thickness of the lines in the headers and footers:

\renewcommand{\headrulewidth}{0 pt}
\renewcommand{\footrulewidth}{0 pt}

% using the biblatex package to create references

\usepackage[style=apa,sorting=nyt, maxcitenames=2, natbib=true,backend=biber]{biblatex}
\addbibresource{references.bib}



% This allows for an additional line-breaking pass with the amount of "tolerable" white space per line increased by 1em.
\emergencystretch=1em



%% end preamble%%

\begin{document}

\begin{titlepage}
   \begin{center}
       \vspace*{1cm}
 
       \textbf{ \begin{LARGE}The Application of Machine Learning\\ 
       \vspace{0.3cm} Methods to Time Series Forecasting \end{LARGE}}
 
       \vspace{0.5cm}
        \begin{large} Improving Forecasting Techniques for Smart City Planning \end{large}
 
       \vspace{1.5cm}
 
       \textbf{Authors:}\\
       
       \begin{center}
	  \begin{tabular}{ c c c } 
   \textbf{Manuel Alexander Schreiber}  &  & \textbf{Conor Hasselgard Cavanaugh}\\
	  (676991) &  & (82502)\\ 
	 \end{tabular}
	 \end{center}

	
	   \vspace{0.5cm}        
        
 	  \textbf{Supervisor:} Prof. Lisbeth La Cour, PhD \\
 	  (Department of Economics)
 	  
 	  \vspace{0.5cm}    
 	  
 	  \textbf{Co-Supervisor:} Prof. Raghava Rao Mukkamala, PhD \\
 	  (Department of Digitalization\\ 
 	  and Centre for Business Data Analytics)
 
       \vspace{3cm}
 	
 	A thesis presented for the degrees of a\\
 	     \begin{center}
	  	 \begin{tabular}{ c c c } 
   Master of Science (M.Sc.)  &  & Cand.Merc.Mat (M.Sc.)\\
    in Advanced Economics and Finance &  & in Erhvervsøkonomi og Matematik\\ 
		 \end{tabular}
	 	 \end{center}
	 
 
 
       \vspace{1cm}
 
       \includegraphics[width=0.4\textwidth]{CBS}
 
       \vspace{1cm}
       
       Number of pages (characters):
       
       Copenhagen Business School\\
       Copenhagen, June 30, 2020
 
   \end{center}
\end{titlepage}

\thispagestyle{plain}
\begin{center}
    \textbf{Abstract}
\end{center}

% type abstract here:

\doublespacing
\tableofcontents
\singlespacing

\thispagestyle{plain}
\clearpage

\listoffigures

\clearpage

\listoftables

% displays acronyms in boldface
\renewcommand{\glsnamefont}[1]{\textbf{#1}}
% changes the vertical space between two entries
\renewcommand*{\arraystretch}{1.3}
% prints the glossary (list of acronyms)
\printnoidxglossary[type=\acronymtype, sort = letter, title = List of Acronyms, nonumberlist, style = long]

\clearpage

% When writing something like a thesis its worth splitting up the document into % multiple .tex files. It's also wise to organise the project using folders;  %therefore, we'll create two new folders, one for all the images used in the %project and one for all the .tex files making up the main body of the thesis.

%\chapter*{Abstract}
%Abstract goes here

\chapter*{Acknowledgements}
I want to thank my supervisor/parents/classmates...

\chapter{Introduction}
Machine learning methods have been developed and refined to the point that they now pose a serious challenge to classic statistical models in the area of forecasting. For example, \citeauthor*{ahmed} \parencite*{ahmed} came up with a large-scale comparison study based on the M3 competition dataset to compare the major machine learning models for time series forecasting. Exponential smoothing models are among the most frequently used forecasting methods. However, \textcite{bergmeir} argue that these models can be substantially outperformed for the purpose of forecasting by applying the bagging algorithm to a bootstrapped component of the initial time series to estimate an ensemble of exponential smoothing models and by combining the resulting point forecasts. 

\section{Preliminary Overview of Relevant Literature}

\citeauthor*{januschowski2020} \parencite*{januschowski2020} provide an overview of the spectrum of Machine learning and statistical methods and the boundaries and intersections in the context of forecasting. This distinction is a common way of classifying forecasting methods in the forecasting literature. However, the authors contend that there is no added value from such a classification of the methods and they encourage the scientific communities to collaborate more.
\textcite{maasoumi2010} present applications in Economics and Finance at the intersection of Statistical Learning and Econometrics. Inter alia, the authors analyze bagging and combination forecasts and find that bagging forecasts often deliver the lowest mean squared forecast errors.

\section{Planned Methodology and Research Design}

As suggested by \textcite{bergmeir}, a Box-Cox transformation can be applied to the series in order to stabilize the variance of the time series and subsequently an STL decomposition is employed to break the time series down into the trend, the seasonal part and the remainder. Bootstrapping the remainder of the series is done to achieve stationarity. Furthermore, \citeauthor*{galicia2019} \parencite*{galicia2019} suggest the use of an ensemble time series forecasting model consisting of three components, which are decision trees, gradient boosted trees and Random Forests. For this model, the ensemble weights are computed by weighted least squares. They found that the ensemble outperforms its individual components. The three ensemble components can thus be employed for the purpose of this paper.



\chapter{Research Questions}
This paper is concerned about the following reseach questions:

\begin{enumerate}
\item[\textbf{(Q1)}] How can forecasting methods be employed to improve city infrastructure planning in a smart city?
\item[\textbf{(Q2)}] What are the best forecasting models for the three Smart City usecases traffic, healthcare and security?
\item[\textbf{(Q3)}] Can the forecasting performance of traditional statistical models for time series forecasting be outperformed or complemented by Machine Learning methods?
\end{enumerate}

B
 
\chapter{Smart City}
\input{chapters/chapter03}
 
\chapter{Forecasting Methods}
\section{Traditional Statistical models}

\subsection{ARIMA models}

\subsection{Exponential Smoothing models}

\subsection{Holt-Winters models}

\section{Machine Learning Methods}

\subsection{Bagging for Time Series}

According to \textcite{hastie2009} the bagging estimate is defined by

\begin{equation} \label{eq: bagging}
\hat{f}_{bag}(x)=\frac{1}{B} \sum_{b=1}^B\hat{f}^{*b}(x)
\end{equation}

\noindent The bagging estimate is obtained by averaging the predictions $\hat{f}^{*b}(x)$ for\\ $ b=1,..,B $ from the statistical learning models fitted to a collection of B bootstrap samples obtained from the single training data set via taking repeated samples with replacement, i.e. bootstrapping.\\
Bagging is therefore a general-purpose technique for reducing the variance of a statistical learning method, which also increases that method's prediction accuracy \citep{james2013}.\\

\noindent The well-established bagging method was first introduced by \textcite{breiman1996} but it was not successfully applied in a time series forecasting context until 2016.\\
According to \textcite{petropoulos2018} the complication with time series consists in accounting for non-stationarity and autocorrelation in the bootstrapping procedure in order to produce bootstrapped samples that resemble the original data.\\
\textcite{bergmeir} propose a bootstrapping procedure for time series as illustrated in \cref{fig: Bootstrapping procedure TS} that includes a Box-Cox transformation in order to stabilize the variance and a decomposition either in form of the loess method to extract trend and remainder in case of a non-seasonal time series or in form of a STL decomposition in order to break a seasonal series down into the trend, seasonal and remainder components. 
The Box-Cox transformation, which was first introduced by \textcite{box1964}, is defined as follows:

\begin{equation} \label{eq: box cox}
	w_{t} =
	\begin{cases}
	log(y_{t}), & \lambda=0 \\
	(y_{t}^\lambda-1)/\lambda, & \lambda \neq0\\
	\end{cases}
\end{equation}

The optimal $\lambda \in [0,1]$ is chosen by dividing the series into subseries of length equal to the seasonality and minimzing the coefficient of variation $s/m^{(1-\lambda)}$ across those subseries, where $s$ stands for the standard deviation and $m$ for the sample mean of the subseries.
The authors then apply a bootstrapping method that allows for autocorrelation, the \acrfull{mbb} as suggested by \textcite{kunsch1989}, to the extracted remainder of the series. In the next step, the series is reconstructed from its structural components, i.e. trend, seasonality, and the bootstrapped remainder. Finally, the Box-Cox transformation is inverted. The whole process is then repeated in order to obtain the bootstrapped series.\\ 


\begin{figure} [h]
\centering
\includegraphics[scale=0.8]{TS}
\caption{Bootstrapping procedure for a time series (adapted from \textcite{petropoulos2018})}
\label{fig: Bootstrapping procedure TS}
\end{figure}


\textcite{bergmeir} find that the ensemble of bagged exponential smoothing models outperforms the regular exponential smoothing model consistently for monthly data on the M3 forecasting competition dataset, which is a common medium of comparison of newly introduced forecasting methods with existing state of the art models.


\subsection{Tree-based models for Time Series}

In general, the purpose of decision tree based methods, which are typically used for regression or classification problems, is to segment the predictor space into a number of simpler regions. The name stems from the fact that the set of splitting rules can be described with a tree \parencite{james2013}. \cref{fig: decision tree and prediction surface} illustrates the tree resulting from the partition of a two-dimensional feature space and also depicts the prediction surface resulting from that tree.\\

\begin{figure} [h]
\centering
\includegraphics[scale=0.6]{DT}
\caption{Decision tree and resulting prediction surface (adapted from \textcite{james2013})}
\label{fig: decision tree  and prediction surface}
\end{figure}

The regions $R_{j}, j=1,...,5 $ are called terminal nodes, the points $t_{i}, i=1,...,4 $ where the predictor space, i.e. the set of possible values for predictors $X_{1},X_{2},...,X_{p}$, is split are defined as internal nodes and the connecting segments are called branches.\\
The goal in the attempt to split the feature space into $J$ distinct and non-overlapping regions $R_{1}, R_{2},...,R_{J}$ is to minimize the \acrfull{rss} in case of a regression tree as given below, where $\hat{y}_{R_{j}}$ is the mean response for the training observations in the $j$-th box \parencite{james2013}:

\begin{equation} \label{eq: RSS}
	RSS = \sum_{j=1}^{J} \sum_{i \in R_{j}} (y_{i}-\hat{y}_{R_{j}})^{2}
\end{equation}

\noindent Another key ingredient of decision trees is the concept of "recursive binary splitting" which refers to a top-down approach to successively splitting the predictor space. Mathematically, the predictor $ X_{j}$ and the cutting point $t$ are chosen such that the split of the predictor space into the regions $\{X|X_{j} < t\}$ and $\{X|X_{j} \geq t\}$ leads to the maximum reduction in the RSS.

\textcite{galicia2019} propose a dynamic ensemble model for big data time series forecasting purposes as illustrated in \cref{fig: Dynamic ensemble model} based on the three base models decision trees, random forests and gradient boosted trees. The ensemble weights are computed by weighted least squares assigning higher weights to more accurate ensemble members according to their past performance.\\
 
\noindent The authors found that the ensemble model outperformed the individual base models on a high sampling frequency dataset for the Spanish electricity market in terms of prediction accuracy as evidenced by a lower \acrfull{mae} and \acrfull{rmse}. Moreover, they found that the dynamic ensemble model could outperform \acrfull{ann} and \acrfull{dl} algorithms when evaluating forecast errors by yielding the lowest MAE and RMSE values on an Australian solar power dataset.

\begin{figure} [h]
\centering
\includegraphics[scale=0.6]{ES}
\caption{Dynamic ensemble model (adapted from \textcite{galicia2019})}
\label{fig: Dynamic ensemble model}
\end{figure}

\subsection{Neural Network Models}

Artificial neural networks (ANNs) refer to forecasting models which are based on mathematical models of the brain and which allow for complex non-linear relationships between a label and features.\\

\noindent Generally, neural networks are composed of neurons which are organized in layers, where the predictors form the inputs in the bottom layer and the forecasts form the outputs in the top layer. Forecasts are obtained by a linear combination of the predictors. The weights attached to those predictors are selected using a learning algorithm which minimizes a cost function, such as the \acrfull{mse} \parencite{hyndman2018}.

\begin{figure} [h]
\centering
\includegraphics[scale=0.5]{NN}
\caption{Multilayer feed-forward network with one hidden layer and three hidden neurons (adapted from \textcite{hyndman2018})}
\floatfoot{\textbf{Notes:} The outputs of the nodes in one layer are inputs of the nodes in the next layer and the inputs to each single node are a weighted linear combination. In the hidden layer, the input for the next layer is first transformed using a non-linear function, such as the sigmoid function, in order to make the neural netwok more robust to extreme outliers.}
\label{fig: Multilayer NN}
\end{figure}

\noindent \cref{fig: Multilayer NN} shows a multilayer feed-forward neural network with a single hidden layer, where the inputs to each node are weighted linear combinations. The addition of a hidden layer makes the neural network non-linear.\\ 
\newpage
The inputs to hidden neuron $j$ in \cref{fig: Multilayer NN} is given by

\begin{equation} \label{eq: inputs hidden neuron}
	z_{j} = b_{j} + \sum_{i=1}^4 w_{i,j}x_{i}
\end{equation}

The parameters $b_{1},...,b_{3}, w_{1,1},...,w_{4,3}$ are learned from the data, where the weights $w_{i,j}$ are initially set to random values and then updated from the observed data. 
According to \textcite{hyndman2018}, the number of hidden layers and the number of neurons contained in the hidden layers is usually determined in advance by cross-validation.

\noindent \textcite{hyndman2018} argue that neural networks are not based on a well-defined stochastic model and therefore the derivation of prediction intervals for forecasts from neural network models is not as trivial as for the previous methods outlined in this paper.\\

The \acrfull{lstmnn} was first introduced by \textcite{hochreiter1997}. The LSTM architecture is shown in \cref{fig: LSTM architecture}. 
\newpage
According to \textcite{ma2015}, a LSTM NN consists of an input layer, a recurrent hidden layer and an output layer. In contrast to a regular ANN, the hidden layer contains a memory block with a memory cell to memorize the temporal state as well as two gates to control the input and output activations into the block. Moreover, a forget gate forms part of the memory block to allow for a reset in case the information flow is out of date.

\begin{figure} [h]
\centering
\includegraphics[scale=1.2]{LSTM}
\caption{LSTM neural network architecture (adapted from \textcite{ma2015})}
\label{fig: LSTM architecture}
\end{figure}

\chapter{Data}
\input{chapters/chapter05}

\chapter{Analyses and Results}
\input{chapters/chapter06}

\chapter{Discussion}
\input{chapters/chapter07}

\chapter{Conclusion}
\input{chapters/conclusion}

\appendix
\chapter{Appendix}
\input{chapters/appendix}


% increase spacing between two entries
\setlength\bibitemsep{0.0\itemsep}

% title option specifies the name of the bibliography
\printbibliography[title = References]

\thispagestyle{plain}



\end{document}

